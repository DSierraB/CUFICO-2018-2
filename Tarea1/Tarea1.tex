\documentclass[10.5pt]{article}

% Spanish characters
\usepackage[utf8]{inputenc}
\usepackage[T1]{fontenc}
% French display
\usepackage[english,spanish]{babel}

\usepackage{lastpage}
%Esto me permite usar el comando "\pageref{LastPage}" en el footer.
\renewcommand{\baselinestretch}{1.6}
% Esto controla el interlineado o espaciado!!!
\usepackage{color}
%\newcommand{\red}[1]{{\color{red} #1}}
\newcommand{\red}[1]{{\color{black} #1}}

%Esto me permite poner hipervínculos:
%\usepackage[pdftex,
%       colorlinks=true,
%       urlcolor=blue,       % \href{...}{...} external (URL)
%       filecolor=green,     % \href{...} local file
%       linkcolor=black,       % \ref{...} and \pageref{...}
%       pdftitle={Papers by AUTHOR},
%       pdfauthor={Your Name},
%       pdfsubject={Just a test},
%       pdfkeywords={test testing testable},
%%       pagebackref,%Esto parece que pone un numerito al lado de la referencia (en la seccion de bibliografia), donde se puede clicar y te lleva al lugar del texto donde se le cita.
%       pdfpagemode=None,
%       bookmarksopen=true]{hyperref}


%The following packages are relics, but I don't want to remove them just in case:
\usepackage{amsmath}
\usepackage{array}
\usepackage{latexsym}
\usepackage{amsfonts}
\usepackage{amsthm}
\usepackage{cite}
\usepackage{setspace}
\usepackage{amssymb}
\usepackage{hyperref}

\usepackage{multicol}
\usepackage{color}
%\usepackage{minipage}

\usepackage{graphicx} % Required for including images
\graphicspath{{figures/}} % Location of the graphics files
\usepackage[font=small,labelfont=bf]{caption} % Required for specifying captions to tables and figures

%The defaults margins are huge, so I'll customize it:
\oddsidemargin  -0.0 in
\textwidth 6.5 in
\textheight 8.7 in
\addtolength{\voffset}{-1cm}

%%%%%%%%%%%%%%%%%%%%%%%% HEADER AND FOOTER %%%%%%%%%%%%%%%%%%%%
\usepackage{fancyhdr}
\pagestyle{fancy}

\fancyhead[L]{Tarea 1}
%\fancyhead[L]{CNRS Competition 01-04}
\fancyhead[R]{Jos\'{e} David Ruiz \'{A}lvarez}
\fancyhead[C]{}
\fancyfoot[C]{\thepage /\pageref{LastPage}}

\newlength\FHoffset
\setlength\FHoffset{0cm}

\addtolength\headwidth{2\FHoffset}
\fancyheadoffset{\FHoffset}

\newlength\FHleft
\newlength\FHright

\setlength\FHleft{1cm}
\setlength\FHright{1cm}

\thispagestyle{empty}
%%%%%%%%%%%%%%%%%%%%%%%% HEADER AND FOOTER %%%%%%%%%%%%%%%%%%%%



\begin{document}

%\begin{center}
\noindent
\begin{minipage}[b]{0.75\linewidth}
{\LARGE\bf Tarea 1}\\ %[1mm]
%\end{center}
%{\Large\bf \emph{}}\\ %[3mm]
%{\Large\bf \emph{connections between LHC and neutrino experiments}}\\ %[3mm]
%{\Large\bf \emph{from neutrons to Higgses}}\\ %[3mm]
\large{Jos\'{e} David Ruiz \'{A}lvarez} \\
\small{\href{mailto:josed.ruiz@udea.edu.co}{josed.ruiz@udea.edu.co}} \\ %[3mm]
%\normalsize{Plaza código: 2017010307, Área: Física de fenomenología de altas energías} \\%[3mm]
\normalsize{Instituto de Física, Facultad de Ciencias Exactas y Naturales} \\%[3mm]
\normalsize{\bf Universidad de Antioquia} \\[8mm]
\today %\\[4mm]
\end{minipage}%
%\end{center}
%\begin{minipage}[b]{0.25\linewidth}
%\centering{\includegraphics[width=4cm]{figures/CMS.png}}\\
%%%%%\includegraphics[width=15cm]{figures/UniandesColombia.jpg}\\
%\end{minipage}

%\begin{center}
%{\bf Palabras clave:} CERN, LHC, CMS, Materia Oscura
%\end{center}

%\doublespacing

La primera tarea del curso consiste en instalar el paquete ROOT de acuerdo a las instrucciones que los desarrolladores proveen para tal efecto: \url{https://root.cern.ch/building-root}. Los entregables para la primera tarea son:

\begin{itemize}
\item Un archivo de texto explicando qué es y qué función cumplen dentro de la instalación el ``cmake'' y ``source''. La explicación no debe ser larga y puede realizarse en unas pocas líneas de texto.
\item Prueba de que han instalado ROOT satisfactoriamente en su computador personal. Para esto sirve por ejemplo un ``pantallazo'' de la terminal mostrando las características de su computador y una sesión de ROOT iniciada.
\item Los entregables deben ser subidos al repositorio central en forma de ``pull request'' en la carpeta Tarea1. Cada estudiante debe crear una carpeta con su nombre donde debe poner los entregables de la tarea.
\end{itemize}

El plazo para entregar la Tarea 1 es hasta el viernes 24 de agosto hasta las 23:59.

%\begin{center}
%\includegraphics[width=0.4\linewidth]{DM_detection.jpg}
%\captionof{figure}{Diagrama ilustrativo donde se muestran los 3 procesos principales que se utilizan para la detección de materia oscura. En el sentido de la línea verde se muestra el proceso de producción de materia oscura en colisionadores, la línea roja corresponde a la detección directa y la línea azul a la detección indirecta.}
%\label{fig:DMdetection}
%\end{center}



%\singlespacing
%\begin{thebibliography}{99}
%
%\end{thebibliography}

\end{document}

%%% Local Variables:
%%%   mode: latex
%%%   mode: flyspell
%%%   ispell-local-dictionary: "spanish"
%%% End:
