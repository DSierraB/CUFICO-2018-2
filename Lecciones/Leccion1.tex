\documentclass[10.5pt]{article}

% Spanish characters
\usepackage[utf8]{inputenc}
\usepackage[T1]{fontenc}
% French display
\usepackage[english,spanish]{babel}

\usepackage{lastpage}
%Esto me permite usar el comando "\pageref{LastPage}" en el footer.
\renewcommand{\baselinestretch}{1.6}
% Esto controla el interlineado o espaciado!!!
\usepackage{color}
%\newcommand{\red}[1]{{\color{red} #1}}
\newcommand{\red}[1]{{\color{black} #1}}

%Esto me permite poner hipervínculos:
%\usepackage[pdftex,
%       colorlinks=true,
%       urlcolor=blue,       % \href{...}{...} external (URL)
%       filecolor=green,     % \href{...} local file
%       linkcolor=black,       % \ref{...} and \pageref{...}
%       pdftitle={Papers by AUTHOR},
%       pdfauthor={Your Name},
%       pdfsubject={Just a test},
%       pdfkeywords={test testing testable},
%%       pagebackref,%Esto parece que pone un numerito al lado de la referencia (en la seccion de bibliografia), donde se puede clicar y te lleva al lugar del texto donde se le cita.
%       pdfpagemode=None,
%       bookmarksopen=true]{hyperref}


%The following packages are relics, but I don't want to remove them just in case:
\usepackage{amsmath}
\usepackage{array}
\usepackage{latexsym}
\usepackage{amsfonts}
\usepackage{amsthm}
\usepackage{cite}
\usepackage{setspace}
\usepackage{amssymb}
\usepackage{hyperref}

\usepackage{multicol}
\usepackage{color}
%\usepackage{minipage}

\usepackage{graphicx} % Required for including images
\graphicspath{{figures/}} % Location of the graphics files
\usepackage[font=small,labelfont=bf]{caption} % Required for specifying captions to tables and figures

%The defaults margins are huge, so I'll customize it:
\oddsidemargin  -0.0 in
\textwidth 6.5 in
\textheight 8.7 in
\addtolength{\voffset}{-1cm}

%%%%%%%%%%%%%%%%%%%%%%%% HEADER AND FOOTER %%%%%%%%%%%%%%%%%%%%
\usepackage{fancyhdr}
\pagestyle{fancy}

\fancyhead[L]{Lecci\'{o}n 1}
%\fancyhead[L]{CNRS Competition 01-04}
\fancyhead[R]{Jos\'{e} David Ruiz \'{A}lvarez}
\fancyhead[C]{}
\fancyfoot[C]{\thepage /\pageref{LastPage}}

\newlength\FHoffset
\setlength\FHoffset{0cm}

\addtolength\headwidth{2\FHoffset}
\fancyheadoffset{\FHoffset}

\newlength\FHleft
\newlength\FHright

\setlength\FHleft{1cm}
\setlength\FHright{1cm}

\thispagestyle{empty}
%%%%%%%%%%%%%%%%%%%%%%%% HEADER AND FOOTER %%%%%%%%%%%%%%%%%%%%



\begin{document}

%\begin{center}
\noindent
\begin{minipage}[b]{0.75\linewidth}
{\LARGE\bf Lecci\'{o}n 1}\\ %[1mm]
%\end{center}
%{\Large\bf \emph{}}\\ %[3mm]
%{\Large\bf \emph{connections between LHC and neutrino experiments}}
\large{Jos\'{e} David Ruiz \'{A}lvarez} \\
\small{\href{mailto:josed.ruiz@udea.edu.co}{josed.ruiz@udea.edu.co}} \\ %[3mm]
%\normalsize{Plaza código: 2017010307, Área: Física de fenomenología de altas energías} \\%[3mm]
\normalsize{Instituto de Física, Facultad de Ciencias Exactas y Naturales} \\%[3mm]
\normalsize{\bf Universidad de Antioquia} \\[8mm]
\today %\\[4mm]
\end{minipage}%
%\end{center}
%\begin{minipage}[b]{0.25\linewidth}
%\centering{\includegraphics[width=4cm]{figures/CMS.png}}\\
%%%%%\includegraphics[width=15cm]{figures/UniandesColombia.jpg}\\
%\end{minipage}

%\begin{center}
%{\bf Palabras clave:} CERN, LHC, CMS, Materia Oscura
%\end{center}

%\doublespacing

\section{Contenido}

Dos grandes vertientes en t\'{e}rminos de problemas f\'{i}sicos: 
\begin{itemize}
\item Análisis estadístico de datos
\item Solución numérica de ecuaciones diferenciales
\end{itemize}

En análisis de datos tendremos como objetivos específicos:
\begin{itemize}
\item Definición de un conjunto de datos 
\item Preparación de un conjunto de datos para hacer una medida
\item Visualización de un conjunto de datos
\item Distribuciones de probabilidad
\item Fit de un conjunto de datos a una función
\item Medición de un observable
\item Error en la medida
\end{itemize}

Con respecto a la solución numérica de ecuaciones diferenciales:
\begin{itemize}
\item Conceptos básicos de análisis numérico
\item Problemas de valor inicial para ecuaciones diferenciales ordinarias
\item Método de Euler
\item Método de Runge-Kutta
\item Errores de redondeo
\item Consistencia, convergencia y estabilidad
\end{itemize}

\section{Evaluación}

\begin{itemize}
\item 20\% seguimiento y tareas: Problemas cortos y ejercicios de programación.
\item 30\% proyecto: Problema físico a resolver en grupos.
\item 50\% dos parciales de igual valor. 
\end{itemize}

\section{Otros}

\begin{itemize}
\item Lenguajes de programación: C(++) y python
\item Paquetes: ROOT, numpy, matplotlib, entre otros.
\item Git, Github
\end{itemize}

\section{Primeros pasos}

Las tareas, seguimiento y proyecto deberán presentarse a través de un {\textbf{Pull request}} al repositorio central del curso: \url{https://github.com/jotadram6/CUFICO-2018-2}. Para empezar a trabajar con git y github es suficiente guiarse por los tutoriales:

\begin{itemize}
\item Acciones básicas en git: \url{http://rogerdudler.github.io/git-guide/}, hasta la sección {\textbf{pushing changes}}.
\item Configurar git: \url{https://help.github.com/articles/set-up-git/}
\item Pasos básicos en github: \url{https://help.github.com/articles/create-a-repo/}, \url{https://guides.github.com/activities/hello-world/}
\item Haciendo una copia paralela personal ({\textbf{Fork}}) de un repositorio existente: \url{https://help.github.com/articles/fork-a-repo/}
\item Para sincronizar un repositorio producto de un fork con respecto al repositorio original: Solo es necesario hacerlo una vez en la copia local del repositorio $\rightarrow$ \url{https://help.github.com/articles/configuring-a-remote-for-a-fork/}; acciones necesarias cada vez que haya que hacer la sincronización \url{https://help.github.com/articles/syncing-a-fork/}.
\end{itemize}

Hola mundo en C++. Primero se debe hacer el siguiente script, Hola.cpp:
\begin{verbatim}
/*
 * Encabezado describiendo el script, los autores y cualquier informacion relevante
 */
#include <iostream>    // Para realizar operaciones de entrada y salida
using namespace std;
 
int main() {                        // Inicio de la funcion main
   cout << "Hola mundo" << endl;  // Imprime en patalla el texto hola mundo
   return 0;                        // Termina la funcion main al retornar un valor
}                                   // Final de la funcion main
\end{verbatim}

Para compilar el script se debe ejecutar en una terminal:
\begin{verbatim}
g++ -o Hola Hola.cpp
\end{verbatim}

Donde ``g++'' es el compilador, ``-o'' indica la opción ``output'' para especificar el nombre del ejecutable de salida, ``Hola'' es el nombre del ejecutable que se va a crear.

Para ejecutar nuestro ejecutable solo debe ejecutarse el siguiente comando desde la terminal:
\begin{verbatim}
./Hola
\end{verbatim}

{\textbf{Ejercicio 1:}} Remueva la línea ``using namespace std;'' de Hola.cpp. ¿Qué pasa cuando intentamos compilar?, ¿Cómo podemos arreglarlo?

{\textbf{Ejercicio 2:}} Después de terminar el ejercicio 1, haga un pull request con su script Hola.cpp al repositorio central del curso en la carpeta Ejercicio\_Leccion1.

%\begin{center}
%\includegraphics[width=0.4\linewidth]{DM_detection.jpg}
%\captionof{figure}{Diagrama ilustrativo donde se muestran los 3 procesos principales que se utilizan para la detección de materia oscura. En el sentido de la línea verde se muestra el proceso de producción de materia oscura en colisionadores, la línea roja corresponde a la detección directa y la línea azul a la detección indirecta.}
%\label{fig:DMdetection}
%\end{center}



%\singlespacing
%\begin{thebibliography}{99}
%
%\end{thebibliography}

\end{document}

%%% Local Variables:
%%%   mode: latex
%%%   mode: flyspell
%%%   ispell-local-dictionary: "spanish"
%%% End:
