\documentclass[10.5pt]{article}

% Spanish characters
\usepackage[utf8]{inputenc}
\usepackage[T1]{fontenc}
% French display
\usepackage[english,spanish]{babel}

\usepackage{lastpage}
%Esto me permite usar el comando "\pageref{LastPage}" en el footer.
\renewcommand{\baselinestretch}{1.6}
% Esto controla el interlineado o espaciado!!!
\usepackage{color}
%\newcommand{\red}[1]{{\color{red} #1}}
\newcommand{\red}[1]{{\color{black} #1}}

%Esto me permite poner hipervínculos:
%\usepackage[pdftex,
%       colorlinks=true,
%       urlcolor=blue,       % \href{...}{...} external (URL)
%       filecolor=green,     % \href{...} local file
%       linkcolor=black,       % \ref{...} and \pageref{...}
%       pdftitle={Papers by AUTHOR},
%       pdfauthor={Your Name},
%       pdfsubject={Just a test},
%       pdfkeywords={test testing testable},
%%       pagebackref,%Esto parece que pone un numerito al lado de la referencia (en la seccion de bibliografia), donde se puede clicar y te lleva al lugar del texto donde se le cita.
%       pdfpagemode=None,
%       bookmarksopen=true]{hyperref}


%The following packages are relics, but I don't want to remove them just in case:
\usepackage{amsmath}
\usepackage{array}
\usepackage{latexsym}
\usepackage{amsfonts}
\usepackage{amsthm}
\usepackage{cite}
\usepackage{setspace}
\usepackage{amssymb}
\usepackage{hyperref}

\usepackage{multicol}
\usepackage{color}
%\usepackage{minipage}

\usepackage{graphicx} % Required for including images
\graphicspath{{figures/}} % Location of the graphics files
\usepackage[font=small,labelfont=bf]{caption} % Required for specifying captions to tables and figures

%The defaults margins are huge, so I'll customize it:
\oddsidemargin  -0.0 in
\textwidth 6.5 in
\textheight 8.7 in
\addtolength{\voffset}{-1cm}

%%%%%%%%%%%%%%%%%%%%%%%% HEADER AND FOOTER %%%%%%%%%%%%%%%%%%%%
\usepackage{fancyhdr}
\pagestyle{fancy}

\fancyhead[L]{Lecci\'{o}n 1}
%\fancyhead[L]{CNRS Competition 01-04}
\fancyhead[R]{Jos\'{e} David Ruiz \'{A}lvarez}
\fancyhead[C]{}
\fancyfoot[C]{\thepage /\pageref{LastPage}}

\newlength\FHoffset
\setlength\FHoffset{0cm}

\addtolength\headwidth{2\FHoffset}
\fancyheadoffset{\FHoffset}

\newlength\FHleft
\newlength\FHright

\setlength\FHleft{1cm}
\setlength\FHright{1cm}

\thispagestyle{empty}
%%%%%%%%%%%%%%%%%%%%%%%% HEADER AND FOOTER %%%%%%%%%%%%%%%%%%%%



\begin{document}

%\begin{center}
\noindent
\begin{minipage}[b]{0.75\linewidth}
{\LARGE\bf Lecci\'{o}n 2}\\ %[1mm]
%\end{center}
%{\Large\bf \emph{}}\\ %[3mm]
%{\Large\bf \emph{connections between LHC and neutrino experiments}}\\ %[3mm]
%{\Large\bf \emph{from neutrons to Higgses}}\\ %[3mm]
\large{Jos\'{e} David Ruiz \'{A}lvarez} \\
\small{\href{mailto:josed.ruiz@udea.edu.co}{josed.ruiz@udea.edu.co}} \\ %[3mm]
%\normalsize{Plaza código: 2017010307, Área: Física de fenomenología de altas energías} \\%[3mm]
\normalsize{Instituto de Física, Facultad de Ciencias Exactas y Naturales} \\%[3mm]
\normalsize{\bf Universidad de Antioquia} \\[8mm]
\today %\\[4mm]
\end{minipage}%
%\end{center}
%\begin{minipage}[b]{0.25\linewidth}
%\centering{\includegraphics[width=4cm]{figures/CMS.png}}\\
%%%%%\includegraphics[width=15cm]{figures/UniandesColombia.jpg}\\
%\end{minipage}

%\begin{center}
%{\bf Palabras clave:} CERN, LHC, CMS, Materia Oscura
%\end{center}

%\doublespacing


\section{Conceptos básicos de C++}

\subsection{Operadores}
Operadores binarios: Requieren dos variables, por ejemplo para realizar la operación suma $a+b$.

\begin{itemize}
\item Suma: +
\item Resta: -
\item Multiplicación: $*$
\item División: $/$
\item Módulo: \% (Solo se puede usar sobre enteros)
\end{itemize}

%Operadores unitarios: Solo requieren de una variable, la operación se efectúa sobre la misma variable. Por ejemplo, $a =+ 10$.

Operadores relacionales, ejemplo $i<=3$, estos operadores retornan un booleano $True$ o $False$ si la realción se cumple o no correspondientemente:
\begin{itemize}
\item Menor que: $<$
\item Menor o igual que: $<=$
\item Mayor que: $>$
\item Mayor o igual que: $>=$
\item Igual que: $==$
\item No igual que: $!=$
\end{itemize}

Operadores lógicos, se usan para combinar operaciones relacionales:
\begin{itemize}
\item OR: $||$
\item AND: $\&\&$
\item Negación: $*$
\end{itemize}

Otros operadores:
\begin{itemize}
\item $A+=5\leftrightarrow A=A+5$
\item $A-=5\leftrightarrow A=A-5$
\item $A*=5\leftrightarrow A=A*5$
\item $A/=5\leftrightarrow A=A/5$
\item $A\% =5\leftrightarrow A=A\% 5$
\item $A++\leftrightarrow A=A+1$
\item $A--\leftrightarrow A=A-1$
\end{itemize}

\subsection{Tipos de variables}

Esta es una de las diferencias fundamentales con Python en donde no se requiere declarar el tipo de una variable para asignarles un valor. En C++ es necesario declarar el tipo de una variable para asignarle un valor, sin embargo ambas acciones se pueden hacer en una sola línea. Los tipos de variables básicos (y más usados, pero hay más!) en C++ son:

\begin{itemize}
\item double: Número real de 15 dígitos + signo 
\item float: Número real de 7 dígitos + signo
\item int: Número entero de 10 dígitos + signo
\item char: Un solo caracter.
\item bool: Booleano ``True'' o ``False''
\end{itemize}

El manejo de strings en C++ es poco intuitivo. Si necesitan hacer un programa para hacer algún tipo de operaciones sobre strings es aconsejable usar python en lugar de C++.

\subsection{Bucles y condicionales}

Ejemplo de condicional simple:
\begin{verbatim}
int a = 12;
if (a<=30){
cout << a*5 << endl;
}
\end{verbatim}

Ejemplo de condicional y ``else'':
\begin{verbatim}
int a = 12;
int b = 13;
if (a<=30 && b>30){
cout << "Both conditions satisfied" << endl;
}
else {
cout << "No condition satisfied" << endl;

}
\end{verbatim}

Ejemplo de condicionales anidados:
\begin{verbatim}
float a = 12.53;
if (a<=2){
cout << "First condition satisfied" << endl;
}
else if (a<=6 && a>2){
cout << "Second condition satisfied" << endl;
}
else if (a<=13 && a>6){
cout << "Third condition satisfied" << endl;
}
\end{verbatim}

Ejemplo de ``while'':
\begin{verbatim}
double b = 1.001;
while (b<10000)
{
b*=1.001;
cout << b << endl; 
}
\end{verbatim}

{\textbf{Ejercicio 1:}} Implemente el mismo ejemplo de while de C++ en python y compare el tiempo que toma hacer el cálculo en ambos lenguajes. Suba los resultados a github en forma de pull request en la carpeta ``Ejercicio1\_Leccion2''.  Recuerde poner su nombre en el nombre de los archivos.

%Results on C++
%real    0m0.202s
%user    0m0.051s
%sys     0m0.064s

%Results on python
%real    0m0.211s
%user    0m0.057s
%sys     0m0.084s

Ejemplo de ``for'':
\begin{verbatim}
  double b = 1.001;
  double c = 1.001;
  for (int i=0; i<20; i++)
    {
      b*=(c+i);
      cout << b << endl;
    }
\end{verbatim}

{\textbf{Ejercicio 2:}} Haga un script de C++ que calcule los números de fibonacci y que imprima en pantalla únicamente los números pares de la serie. Guarde los resultados en un archivo de texto. Suba el script y los resultados a github en forma de pull request en la carpeta ``Ejercicio2\_Leccion2''. Recuerde poner su nombre en el nombre de los archivos.


%\begin{center}
%\includegraphics[width=0.4\linewidth]{DM_detection.jpg}
%\captionof{figure}{Diagrama ilustrativo donde se muestran los 3 procesos principales que se utilizan para la detección de materia oscura. En el sentido de la línea verde se muestra el proceso de producción de materia oscura en colisionadores, la línea roja corresponde a la detección directa y la línea azul a la detección indirecta.}
%\label{fig:DMdetection}
%\end{center}



%\singlespacing
%\begin{thebibliography}{99}
%
%\end{thebibliography}

\end{document}

%%% Local Variables:
%%%   mode: latex
%%%   mode: flyspell
%%%   ispell-local-dictionary: "spanish"
%%% End:
